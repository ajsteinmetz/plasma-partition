\documentclass[aps,prd,floatfix,reprint]{revtex4-2}
% Leave uncommented if the LaTeX file is uploaded to arXiv.org
\pdfoutput=1
\pdfminorversion=7
\bibliographystyle{apsrev4-2}

% Packages
%\usepackage{arxiv}
\usepackage[colorlinks=true,linkcolor=cyan,citecolor=cyan]{hyperref}
%\usepackage[numbers]{natbib}
%\usepackage{authblk}
%\usepackage{caption}
%\usepackage{subcaption}
\usepackage{graphicx}
%\usepackage{dcolumn}
\usepackage{bm}
\usepackage{amsmath}
\usepackage{amssymb}
\usepackage{epstopdf}
\usepackage{comment}
\usepackage{xcolor}
\usepackage{multirow}
%\usepackage{float}
%\usepackage{doi}
%\usepackage[mathlines]{lineno}% Enable numbering of text and display math
%\modulolinenumbers[5]% Line numbers with a gap of 5 lines
%\linenumbers\relax % Commence numbering lines

% Make Orcid icon
\newcommand{\orcidicon}{\includegraphics[width=0.32cm]{orcid.pdf}}
\newcommand{\orc}[1]{\href{https://orcid.org/#1}{\orcidicon}}

% Author Orcid ID: Define per author
\newcommand{\orcA}{0000-0001-8217-1484}
\newcommand{\orcB}{0000-0001-5038-8427}
\newcommand{\orcC}{0000-0001-5474-2649}

% Useful macros for equations and units in HEP
\newcommand*{\TeV}{\text{ TeV}}
\newcommand*{\GeV}{\text{ GeV}}
\newcommand*{\MeV}{\text{ MeV}}
\newcommand*{\keV}{\text{ keV}}
\newcommand*{\eV}{\text{ eV}}
\newcommand*{\meV}{\text{ meV}}
\newcommand*{\bb}{\boldsymbol}
\newcommand*{\beqn}{\begin{equation}}
\newcommand*{\eeqn}{\end{equation}}
\newcommand{\req}[1]{Eq.\,(\ref{#1})}
\newcommand{\rf}[1]{Fig.~{\ref{#1}}}
\newcommand{\rt}[1]{Table~{\ref{#1}}}
\newcommand{\rsec}[1]{Sect.\,{\ref{#1}}}

% Useful macros for annotation
\newcommand*{\xred}{\color{red}}
\newcommand*{\xblue}{\color{blue}}
\newcommand*{\xgreen}{\color{green}}

 
\begin{document}

\title{Matter-antimatter origin of cosmic magnetism: A first look}
\author{Andrew Steinmetz\orc{\orcC}}
\author{Cheng Tao Yang\orc{\orcB}}
\author{Johann Rafelski\orc{\orcA}}
\affiliation{Department of Physics, The University of Arizona, Tucson, AZ 85721, USA}

%\date{\today}
\date{August 20, 2023}

\begin{abstract}
We explore the hypothesis that the abundant presence of relativistic antimatter (positrons) in the primordial universe is responsible for the inter-galactic magnetic fields we observe in the universe today. We evaluate the magnetic properties of the primordial electron-positron $e^{+}e^{-}$ plasma, and obtain in quantitative terms the magnitude of the relatively small $e^{+}e^{-}$ polarization asymmetry required to produce a consistent self-magnetization.
\end{abstract}

\keywords{early universe cosmology, magnetization, electron-positron plasma, intergalactic magnetic fields}

\maketitle

%%%%%%%%%%%%%%%%%%%%%%%%%%%%%%%%%%%%%%%
\section{Introduction}
\label{sec:introduction}
%%%%%%%%%%%%%%%%%%%%%%%%%%%%%%%%%%%%%%%
\noindent Macroscopic domains of magnetic fields have been found around compact objects (stars, planets, etc.); between stars; within galaxies; between galaxies in clusters; and surprisingly in deep extra-galactic void spaces. Considering this ubiquity of magnetic fields in the universe~\cite{Giovannini:2017rbc,Giovannini:2003yn,Kronberg:1993vk}, we search for a common primordial mechanism.

We investigate the hypothesis that the observed intergalactic magnetic fields (IGMF) predate the recombination epoch. In this work, IGMF will refer to experimentally observed intergalactic fields of any origin while primordial magnetic fields (PMF) refers to fields generated via early universe processes possibly as far back as inflation.

We consider in this work a novel hypothesis that the magnetization of the early universe is driven by relativistic spin paramagnetism originating in the dense electron-positron $e^{+}e^{-}$ plasma shown in \rf{fig:densityratio} in terms of the antimatter (positron) abundance ratio to the prevailing baryon density. Our hypothesis coincidentally occurs around the era of Big Bang Nucleosynthesis (BBN). Electrons and positrons (having the largest magnetic moments in nature) in the pre-BBN early universe could induce a magnetic field in presence of a tiny spin orientation polarization. Since electrons and positrons have opposite magnetic moments, the induced magnetization does not entail a net local spin density statistical average.

Moreover, the required polarization is tiny at high $T$ since the $e^{+}e^{-}$ pair abundance was nearly 450 million pairs per baryon above temperature $T>m_ec^2=511\keV$ dropping to about 100 million pairs per baryon at the pre-BBN temperature $T=100\keV$. Antimatter therefore dominated not only numerically over the baryon content, but also in terms of energy density and even in terms of mass density. Massive dark matter is a negligible component in the Universe in this considered epoch; see Fig.\,2 in Ref.~\cite{Rafelski:2023emw}. The abundance of $e^{+}e^{-}$  is discussed in more detail in \rsec{sec:abundance} where more details leading to results shown in \rf{fig:densityratio} are also presented. Pair abundance fully disappears below  $T=21\keV$ long after the BBN era ended. 

%%%%%%%%%%%%%%%%%%%%%%%%%%%%%%%%%%%%%%%
\begin{figure}[ht]
 \centering
\includegraphics[width=0.48\textwidth]{plots/EEPlasmaDensityRatio_new01.jpg}
 \caption{Number density of electron $e^{-}$ and positron $e^{+}$ to baryon ratio $n_{e^{\pm}}/n_{B}$ as a function of photon temperature in the universe. See text in \rsec{sec:abundance} for further details. In this work we measure temperature in units of energy (keV) thus we set the Boltzmann constant to $k_{B}=1$. We consider all results in temporal sequence in the expanding universe, thus we begin with high temperatures and early times on the left and end at lower temperatures and later times on the right.}
 \label{fig:densityratio} 
\end{figure}
%%%%%%%%%%%%%%%%%%%%%%%%%%%%%%%%%%%%%%%

Our analysis in \rsec{sec:thermal} of the relativistic fermion partition function focuses on the spin contribution to magnetization. At $T\simeq200\keV$, due to very high $e^{+}e^{-}$ pair densities, we believe that spin paramagnetism is dominant over Landau orbital diamagnetism. We show in \rsec{sec:magnetization} that magnetization is nonzero even for a nearly symmetric particle-antiparticle gas as well as account for the matter-antimatter asymmetry present in the universe. Our description of relativistic paramagnetism is covered in \rsec{sec:paramagnetism}. We further demonstrate in \rsec{sec:ferro} that magnetization can be spontaneously increased in strength near the IGMF upper limit seen in \req{igmf} given sufficient polarization.

We do not address here the origin of the polarization phenomena but recall that the $e^{+}e^{-}$ pair plasma decouples from the neutrino background near to $T=2000\keV$~\cite{Birrell:2014uka}. Therefore
we consider the magnetic properties of the $e^{+}e^{-}$ pair plasma in the temperature range  $2000\keV>T>21\keV$ and focus on the range $200\keV>T>21\keV$ where the most rapid antimatter abundance changes occurs and where the Boltzmann approximation is valid. This is notably the final epoch where antimatter exists in large quantities in the cosmos~\cite{Rafelski:2023emw}. 

This is assisted by the fact that antiparticles $(e^{+})$ have the opposite sign of charge, and thus magnetic moment, compared to particles $(e^{-})$. Therefore in an $e^{+}e^{-}$ pair plasma, net magnetization can be associated with opposite spin orientations for particles and antiparticles without the accompaniment of a net angular momentum in the volume considered. This is of course very different from the the matter dominated universe arising below $T\simeq20\keV$ which includes the current epoch. Our findings are summarized in \rsec{sec:conclusions} and further research avenues are commented upon.

As the Universe evolved, the primordial magnetization due to relativistic spin polarization has further dynamical consequences which are beyond the scope of this presentation: The rapid $10^{8}$ drop in $e^{+}e^{-}$ abundance within the temperature range $100\keV>T>21\keV$ shown in \rf{fig:densityratio} could be inducing dynamical currents preserving (comoving) magnetic flux in the emerging $p,\alpha,e^{-}$ plasma. Large primordial polarization sourced magnetic fields could therefore lead to plasma flow maintaining fields. Such rotational dynamics, baryon inhomogeneities could in turn produce anisotropies in the cosmic microwave background (CMB)~\cite{Jedamzik:2013gua,Abdalla:2022yfr}.

%%%%%%%%%%%%%%%%%%%%%%%%%%%%%%%%%%%%%%%
\subsection{Short survey of observational cosmic magnetism}
\label{sec:survey}
%%%%%%%%%%%%%%%%%%%%%%%%%%%%%%%%%%%%%%%
\noindent Here we detail some of the efforts to observe and explain large scale cosmic magnetic fields which motivates our work. IGMFs are notably difficult to measure and difficult to explain. The bounds for IGMF at a length scale of $1{\rm\ Mpc}$ are today~\cite{Neronov:2010gir,Taylor:2011bn,Pshirkov:2015tua,Jedamzik:2018itu,Vernstrom:2021hru}
\begin{align}
 \label{igmf}
 10^{-8}{\rm\ G}>\mathcal{B}_{\rm IGMF}>10^{-16}{\rm\ G}\,.
\end{align}
We note that generating PMFs with such large coherent length scales is nontrivial~\cite{Giovannini:2022rrl} though currently the length scale for PMFs are not well constrained either~\cite{AlvesBatista:2021sln}. The conventional elaboration of the origins for cosmic PMFs are detailed in~\cite{Gaensler:2004gk,Durrer:2013pga,AlvesBatista:2021sln}.

Faraday rotation from distant radio active galaxy nuclei (AGN)~\cite{Pomakov:2022cem} suggest that neither dynamo nor astrophysical processes would sufficiently account for the presence of magnetic fields in the universe today if the IGMF strength was around the upper bound of ${\cal B}_{\rm IGMF}\simeq30-60{\rm\ nG}$ as found in Ref.~\cite{Vernstrom:2021hru}. Such strong magnetic fields would then require that at least some portion of the IGMF arise from primordial sources that predate the formation of stars. Ref.~\cite{Jedamzik:2020krr} propose further that the presence of a magnetic field of ${\cal B}_{\rm PMF}\simeq0.1{\rm\ nG}$ could be sufficient to explain the Hubble tension. 

%%%%%%%%%%%%%%%%%%%%%%%%%%%%%%%%%%%%%%%
\section{Electron-positron abundance}
\label{sec:abundance}
%%%%%%%%%%%%%%%%%%%%%%%%%%%%%%%%%%%%%%%
\noindent As the universe cooled below temperature $T=m_{e}$ (the electron mass), the thermal electron and positron comoving density depleted by over eight orders of magnitude. At $T_{\rm split}=20.3\keV$, the charged lepton asymmetry (mirrored by baryon asymmetry and enforced by charge neutrality) became evident as the surviving excess electrons persisted while positrons vanished entirely from the particle inventory of the universe due to annihilation.

The electron-to-baryon density ratio $n_{e^{-}}/n_{B}$ is shown in \rf{fig:densityratio} as the solid blue line while the positron-to-baryon ratio $n_{e^{+}}/n_{B}$ is represented by the dashed red line. These two lines overlap until the temperature drops below $T_{\rm split}=20.3\keV$ as positrons vanish from the universe marking the end of the $e^{+}e^{-}$ plasma and the dominance of the electron-proton $(e^{-}p)$ plasma. The two vertical dashed green lines denote temperatures $T=m_{e}\simeq511\keV$ and $T_{\rm split}=20.3\keV$. These results were obtained using charge neutrality and the baryon-to-photon content (entropy) of the universe; see details in ~\cite{Rafelski:2023emw}. The two horizontal black dashed lines denote the relativistic $T\gg m_e$ abundance of $n_{e^{\pm}}/n_{B}=4.47\times10^{8}$ and post-annihilation abundance of $n_{e^{-}}/n_{B}=0.87$. Above temperature $T\simeq85\keV$, the $e^{+}e^{-}$ primordial plasma density exceeded that of the Sun's core density $n_{e}\simeq6\times10^{26}{\rm\ cm}^{-3}$~\cite{Bahcall:2000nu}. 

Conversion of the dense $e^{+}e^{-}$ pair plasma into photons reheated the photon background~\cite{Birrell:2014uka} separating the photon and neutrino temperatures. The $e^{+}e^{-}$ annihilation and photon reheating period lasted no longer than an afternoon lunch break. Because of charge neutrality, the post-annihilation comoving ratio $n_{e^{-}}/n_{B}=0.87$~\cite{Rafelski:2023emw} is slightly offset from unity in~\rf{fig:densityratio} by the presence of bound neutrons in $\alpha$ particles and other neutron containing light elements produced during BBN epoch. 

To obtain a quantitative description of the above evolution, we study the bulk properties of the relativistic charged/magnetic gasses in a nearly homogeneous and isotropic primordial universe via the thermal Fermi-Dirac or Bose distributions. For matter $(\sigma=+1)$ and antimatter $(\sigma=-1)$ particles, a nonzero relativistic chemical potential $\mu_{\sigma}=\sigma\mu$ is caused by an imbalance of matter and antimatter. While the primordial electron-positron plasma era was overall charge neutral, there was a small asymmetry in the charged leptons from baryon asymmetry~\cite{Fromerth:2012fe,Canetti:2012zc} in the universe. Consideration of reactions such as $e^{+}e^{-}\leftrightarrow\gamma\gamma$ constrains the chemical potentials of electrons and positrons~\cite{Elze:1980er} as 
\begin{align}
 \label{cpotential}
 \mu\equiv\mu_{e^{-}}=-\mu_{e^{+}}\,,\qquad
 \lambda\equiv\lambda_{e^{-}}=\lambda_{e^{+}}^{-1}=\exp\frac{\mu}{T}\,,
\end{align}
where $\lambda$ is the fugacity of the system. 

During the $e^{+}e^{-}$ plasma epoch, the density changed dramatically over time (see \rf{fig:densityratio}) changing the chemical potential in turn. We can then parameterize the chemical potential of the $e^{+}e^{-}$ plasma as a function of temperature $\mu\rightarrow\mu(T)$ via the charge neutrality of the universe which implies
\begin{align}
 \label{chargeneutrality}
 n_{p}=n_{e^{-}}-n_{e^{+}}=\frac{1}{V}\lambda\frac{\partial}{\partial\lambda}\ln{\cal Z}_{e^{+}e^{-}}\,.
\end{align}
In \req{chargeneutrality}, $n_{p}$ is the observed total number density of protons in all baryon species. The parameter $V$ relays the proper volume under consideration and $\ln{\cal Z}_{e^{+}e^{-}}$ is the partition function for the electron-positron gas. The chemical potential defined in \req{cpotential} is obtained from the requirement that the positive charge of baryons (protons, $\alpha$ particles, light nuclei produced after BBN) is exactly and locally compensated by a tiny net excess of electrons over positrons.

The abundance of baryons is itself fixed by the known abundance relative to photons~\cite{ParticleDataGroup:2022pth} and we employed the contemporary recommended value $n_B/n_\gamma=6.09\times 10^{-10}$. The resulting chemical potential needs to be evaluated carefully to obtain the behavior near to $T_{\rm split}=20.3\keV$ where the relatively small value of chemical potential $\mu$ rises rapidly so that positrons vanish from the particle inventory of the universe while nearly one electron per baryon remains. The detailed solution of this problem is found in Refs.\;\cite{Fromerth:2012fe,Rafelski:2023emw} leading to the results shown in \rf{fig:densityratio}. These results are obtained allowing for Fermi-Dirac and Bose statistics, however it is often numerically sufficient to consider the Boltzmann distribution limit; see \rsec{sec:paradia}.

In the presence of a magnetic field ${\cal B}$ along a primary axis, there is some modification of the usual relativistic fermion partition function which is now given by
\begin{multline}
 \label{partition:1}
 \ln{\cal Z}_{e^{+}e^{-}}=\frac{e{\cal B}V}{(2\pi)^{2}}\sum_{\sigma}^{\pm1}\sum_{s}^{\pm1}\sum_{n=0}^{\infty}\int_{-\infty}^{\infty}{\rm d}p_{z}\\
 \left[\ln\left(1+\lambda_{\sigma}\xi_{\sigma,s}\exp\left(-\frac{E}{T}\right)\right)\right]\,
\end{multline}
\begin{align}
    \label{partition:2}    
    \lambda_{\sigma}\xi_{s} = \exp{\frac{\mu_{\sigma}+\eta_{\sigma,s}}{T}}\,,
\end{align}
where $p_{z}$ is the momentum parallel to the field axis and electric charge is $e\equiv q_{e^{+}}=-q_{e^{-}}$. The index $\sigma$ in \req{partition:1} is a sum over electron and positron states while $s$ is a sum over polarizations. The index $s$ refers to the spin along the field axis: parallel $(s=+1)$ or anti-parallel $(s=-1)$ for both particle and antiparticle species. The quantum numbers of the energy eigenstate $E$ will be elaborated on in \rsec{sec:thermal}.

We are explicitly interesting in small asymmetries such as baryon excess over antibaryons, or one polarization over another. These are described by \req{partition:2} as the following two fugacities:
\begin{itemize}
 \item[a.] Chemical fugacity $\lambda_{\sigma}$
 \item[b.] Polarization fugacity $\xi_{\sigma,s}$
\end{itemize}
The chemical fugacity $\lambda_{\sigma}$ (defined in \req{cpotential} above) describes deformation of the Fermi-Dirac distribution due to nonzero chemical potential $\mu$. An imbalance in electrons and positrons leads as discussed earlier to a nonzero particle chemical potential $\mu\neq0$. We then introduce a novel polarization fugacity $\xi_{\sigma,s}$ and polarization potential $\eta_{\sigma,s}=\sigma s\eta$. We propose the polarization potential follows analogous expressions as seen in \req{cpotential} obeying
\begin{align}
 \label{spotential}
 \eta\equiv\eta_{+,+}=\eta_{-,-}\,,\quad\eta=-\eta_{\pm,\mp}\,,\quad\xi_{\sigma,s}\equiv\exp{\frac{\eta_{\sigma,s}}{T}}\,.
\end{align}

An imbalance in polarization within a region of volume $V$ results in a nonzero spin potential $\eta\neq0$. Conveniently since antiparticles have opposite sign of charge and magnetic moment, the same magnetic moment is associated with opposite spin orientation for particles and antiparticles independent of degree of spin-magnetization. A completely particle-antiparticle symmetric magnetized plasma will have therefore zero total angular momentum. This is of course very different from the situation today of a matter dominated universe.

%%%%%%%%%%%%%%%%%%%%%%%%%%%%%%%%%%%%%%%
\section{Theory of thermal matter-antimatter plasmas}
\label{sec:thermal}
%%%%%%%%%%%%%%%%%%%%%%%%%%%%%%%%%%%%%%%
\noindent As the universe undergoes isotropic expansion, the temperature decreases adiabatically~\cite{Abdalla:2022yfr} and conserves entropy as 
\begin{align}
 \label{tscale}
 T(t)=T_{0}\frac{a_{0}}{a(t)}\rightarrow T(z)=T_{0}(1+z)\,,
\end{align}
where $a(t)$ is the scale factor defined by the Friedmann-Lema{\^i}tre-Robertson-Walker (FLRW) metric~\cite{weinberg1972gravitation} and $z$ is the redshift. The comoving temperature $T_{0}$ is given by the present day temperature of the CMB, with contemporary scale factor $a_{0}=1$. Within a homogeneous magnetic domain, the magnetic field magnitude varies~\cite{Durrer:2013pga} over cosmic expansion as
\begin{align}
 \label{bscale}
 {\cal B}(t)={\cal B}_{0}\frac{a_{0}^{2}}{a^{2}(t)}\rightarrow{\cal B}(z)={\cal B}_{0}\left(1+z\right)^{2}\,,
\end{align}
where ${\cal B}_{0}$ is the comoving value of the magnetic field obtained from the contemporary value today given in \req{igmf}. Non-primordial magnetic fields (which are generated through other mechanisms such as dynamo or astrophysical sources) do not follow this scaling~\cite{Pomakov:2022cem}. The presence of matter and late universe structure formation also contaminates the primordial field evolution in \req{bscale}. It is only in deep intergalactic space where primordial fields remain preserved and comoving over cosmic time.

From \req{tscale} and \req{bscale} emerges a natural ratio of interest here which is conserved over cosmic expansion 
\begin{gather}
 \label{tbscale}
 b \equiv\frac{e{\cal B}(t)}{T^{2}(t)}=\frac{e{\cal B}_{0}}{T_{0}^{2}}\equiv b_0={\rm\ const.}\\
 10^{-3}>b_{0}>10^{-11}\,,
\end{gather}
given in natural units ($c=\hbar=k_{B}=1$). We computed the bounds for this cosmic magnetic scale ratio by using the present day IGMF observations given by \req{igmf} and the present CMB temperature $T_{0}=2.7{\rm\ K}\simeq2.3\times10^{-4}\eV$~\cite{Planck:2018vyg}.

To evaluate magnetic properties of the thermal $e^{+}e^{-}$ pair plasma we take inspiration from Ch. 9 of Melrose's treatise on magnetized plasmas~\cite{melrose2008quantum}. We focus on the bulk properties of thermalized plasmas in (near) equilibrium. In considering $e^{+}e^{-}$ pair plasma, we introduce the microscopic energy of the charged relativistic fermion within a homogeneous ($z$-direction) magnetic field~\cite{Steinmetz:2018ryf}. The energy eigenvalue is given by
\begin{align}
 \label{kgp}
 E^{n}_{\sigma,s}(p_{z},{\cal B})=\sqrt{m_{e}^{2}+p_{z}^{2}+e{\cal B}\left(2n+1+\frac{g}{2}\sigma s\right)}\,,
\end{align}
where $n\in0,1,2,\ldots$ is the Landau orbital quantum number. \req{kgp} differentiates between electrons and positrons which is to ensure the correct non-relativistic limit is reached; see \rt{tab:1}. The parameter $g$ is the gyro-magnetic ($g$-factor) of the particle. Following the conventions found in \cite{Tiesinga:2021myr}, we set $g\equiv g_{e^{+}}=-g_{e^{-}}>0$ such that electrons and positrons have opposite $g$-factors and opposite magnetic moments.

As statistical properties depend on the characteristic Boltzmann factor $E/T$, another interpretation of \req{tbscale} in the context of energy eigenvalues (such as those given in \req{kgp}) is the preservation of magnetic moment energy relative to momentum under adiabatic cosmic expansion.

We rearrange \req{kgp} by pulling the spin dependency and the ground state Landau orbital into the mass writing
\begin{gather}
 \label{effmass:1}
 E^{n}_{\sigma,s}={\tilde m}_{\sigma,s}\sqrt{1+\frac{p_{z}^{2}}{{\tilde m}_{\sigma,s}^{2}}+\frac{2e{\cal B}n}{{\tilde m}_{\sigma,s}^{2}}}\,,\\
 \label{effmass:2}
 \varepsilon_{\sigma,s}^{n}(p_{z},{\cal B})=\frac{E_{\sigma,s}^{n}}{{\tilde m}_{\sigma,s}}\,,\qquad{\tilde m}_{\sigma,s}^{2}=m_{e}^{2}+e{\cal B}\left(1+\frac{g}{2}\sigma s\right)\,,
\end{gather}
where we introduced the dimensionless energy $\varepsilon^{n}_{\sigma,s}$ and effective polarized mass ${\tilde m}_{\sigma,s}$ which is distinct for each spin alignment and is a function of magnetic field strength ${\cal B}$. The effective polarized mass ${\tilde m}_{\sigma,s}$ allows us to describe the $e^{+}e^{-}$ plasma with the spin effects almost wholly separated from the Landau characteristics of the gas when considering the plasma's thermodynamic properties.

%%%%%%%%%%%%%%%%%%%%%%%%%%%%%%%%%%%%%%%
\begin{figure}[ht]
    \centering
    \includegraphics[width=0.3\textwidth]{plots/schematic.png}
    \begin{tabular}{ r|c|c| }
    \multicolumn{1}{r}{}
     &  \multicolumn{1}{c}{aligned: $s=+1$}
     & \multicolumn{1}{c}{anti-aligned: $s=-1$} \\
    \cline{2-3}
    electron: $\sigma=+1$ & $+|\vec{\mu}\cdot\vec{\cal B}|$ & $-|\vec{\mu}\cdot\vec{\cal B}|$ \\
    \cline{2-3}
    positron: $\sigma=-1$ & $-|\vec{\mu}\cdot\vec{\cal B}|$ & $+|\vec{\mu}\cdot\vec{\cal B}|$ \\
    \cline{2-3}
    \end{tabular}
        \caption{Organizational schematic of matter-antimatter $(\sigma)$ and polarization $(s)$ states with respect to the sign of the magnetic dipole energy.}
    \label{tab:1}
\end{figure}
%%%%%%%%%%%%%%%%%%%%%%%%%%%%%%%%%%%%%%%

Since we address the temperature interval $200\keV>T>20\keV$ where the effects of quantum Fermi statistics on the $e^{+}e^{-}$ pair plasma are relatively small, but the gas is still considered relativistic, we will employ the Boltzmann approximation to the partition function in \req{partition:1}. However, we extrapolate our results for presentation completeness up to $T\simeq 4m_{e}$.

In general, modifications due to quantum statistical phase-space reduction for fermions are expected to suppress results by about 20\% in the extrapolated regions. We will continue to search for semi-analytical solutions for Fermi statistics in relativistic $e^{+}e^{-}$ pair gasses to compliment the Boltzmann solution offered here. 

%%%%%%%%%%%%%%%%%%%%%%%%%%%%%%%%%%%%%%%
\subsection{Spin and spin-orbit partition functions}
\label{sec:paradia}
%%%%%%%%%%%%%%%%%%%%%%%%%%%%%%%%%%%%%%%
\noindent We will proceed in this section with the Boltzmann approximation for the limit where $T\lesssim m_e$. The partition function shown in equation \req{partition:1} can be rewritten removing the logarithm as
\begin{multline}
 \label{partitionpower:1}
 \ln{{\cal Z}_{e^{+}e^{-}}}=\frac{e{\cal B}V}{(2\pi)^{2}}\sum_{\sigma,s}^{\pm1}\sum_{n=0}^{\infty}\sum_{k=1}^{\infty}\int_{-\infty}^{+\infty}{\rm d}p_{z}\\
 \frac{(-1)^{k+1}}{k}\exp\left({k\frac{\sigma\mu+\sigma s\eta-{\tilde m}_{\sigma,s}\varepsilon^{n}_{\sigma,s}}{T}}\right)\,,
\end{multline}
\begin{align}
 \label{bapprox} 
 \sigma\mu+\sigma s\eta-{\tilde m}_{\sigma,s}\varepsilon_{\sigma,s}^{n}<0\,,
\end{align}
which is well behaved as long as the factor in \req{bapprox} remains negative. We evaluate the sums over $\sigma$ and $s$ as
\begin{multline}
    \label{partitionpower:2}
    \ln{{\cal Z}_{e^{+}e^{-}}}=\frac{e{\cal B}V}{(2\pi)^{2}}\sum_{n=0}^{\infty}\sum_{k=1}^{\infty}\int_{-\infty}^{+\infty}{\rm d}p_{z}\frac{(-1)^{k+1}}{k}\\
    \left(\ \exp\left(k\frac{+\mu+\eta}{T}\right)\exp\left(-k\frac{{\tilde m}_{+,+}\varepsilon_{+,+}^{n}}{T}\right)\right.\\
    +\exp\left(k\frac{+\mu-\eta}{T}\right)\exp\left(-k\frac{{\tilde m}_{+,-}\varepsilon_{+,-}^{n}}{T}\right)\\
    +\exp\left(k\frac{-\mu-\eta}{T}\right)\exp\left(-k\frac{{\tilde m}_{-,+}\varepsilon_{-,+}^{n}}{T}\right)\\
    +\left.\exp\left(k\frac{-\mu+\eta}{T}\right)\exp\left(-k\frac{{\tilde m}_{-,-}\varepsilon_{-,-}^{n}}{T}\right)\right)
\end{multline}
We note from \rt{tab:1} that the first and forth terms and the second and third terms share the same energies via
\begin{align}
    \label{partitionpower:3}
    \varepsilon_{+,+}^{n}=\varepsilon_{-,-}^{n}\,,\qquad
    \varepsilon_{+,-}^{n}=\varepsilon_{-,+}^{n}\,.\qquad
    \varepsilon_{+,-}^{n}<\varepsilon_{+,+}^{n}\,,
\end{align}

\req{partitionpower:3} allows us to reorganize the partition function with a new magnetization quantum number $s'$ which characterizes paramagnetic flux increasing states $(s'=+1)$ and diamagnetic flux decreasing states $(s'=-1)$. This recasts \req{partitionpower:2} as
\begin{multline}
    \label{partitionpower:4}
    \ln{{\cal Z}_{e^{+}e^{-}}}=\frac{e{\cal B}V}{(2\pi)^{2}}\sum_{s'}^{\pm1}\sum_{n=0}^{\infty}\sum_{k=1}^{\infty}\int_{-\infty}^{+\infty}{\rm d}p_{z}\frac{(-1)^{k+1}}{k}\\
    \left[2\xi_{s'}\cosh\frac{k\mu}{T}\right]\exp\left(-k\frac{{\tilde m}_{s'}\varepsilon_{s'}^{n}}{T}\right)
\end{multline}
with dimensionless energy, polarization mass, and polarization redefined in terms of $s'$
\begin{gather}
    {\tilde m}_{s'}^{2}=m_{e}^{2}+e{\cal B}\left(1-\frac{g}{2}s'\right)\,,\\
    \eta\equiv\eta_{+}=-\eta_{-}\qquad\xi\equiv\xi_{+}=\xi_{-}^{-1}\,.
\end{gather}

We introduce the modified Bessel function $K_{\nu}$ (see Ch. 10 of~\cite{Letessier:2002ony}) of the second kind
\begin{multline}
 \label{besselk}
 K_{\nu}\left(\frac{m}{T}\right)=\frac{\sqrt{\pi}}{\Gamma(\nu-1/2)}\frac{1}{m}\left(\frac{1}{2mT}\right)^{\nu-1}\\
 \int_{0}^{\infty}{\rm d}p\,p^{2\nu-2}\exp\left({-\frac{m\varepsilon}{T}}\right)\,,
\end{multline}
\begin{align}
 \nu>1/2\,,\qquad\varepsilon=\sqrt{1+p^{2}/m^{2}}\,,
\end{align}
allowing us to rewrite the integral over momentum in \req{partitionpower:4} as
\begin{align}
 \label{besselkint}
 \frac{1}{T}\int_{0}^{\infty}\!\!{\rm d}p_{z}\exp\!\left(\!{-\frac{k{\tilde m}_{s'}\varepsilon_{s'}^{n}}{T}}\!\right)\!=\!W_{1}\!\!\left(\frac{k{\tilde m}_{s'}\varepsilon_{s'}^{n}(0,{\cal B})}{T}\right)\,.
\end{align}
The function $W_{\nu}$ serves as an auxiliary function of the form $W_{\nu}(x)=xK_{\nu}(x)$. The notation $\varepsilon(0,{\cal B})$ in \req{besselkint} refers to the definition of dimensionless energy found in \req{effmass:2} with $p_{z}=0$.

The standard Boltzmann distribution is obtained by summing only $k=1$ and neglecting the higher order terms. The Euler-Maclaurin formula~\cite{abramowitz1988handbook} is used to convert the summation over Landau levels into an integration given by
\begin{multline}
 \label{eulermaclaurin}
 \sum_{n=0}^{\infty}W_{1}(n)=\int_{0}^{\infty}{\rm d}n\,W_{1}(n)+\frac{1}{2}\left[W_{1}(\infty)+W_{1}(0)\right]\\
 +\frac{1}{12}\left[\frac{\partial W_{1}}{\partial n}\bigg\vert_{\infty}-\frac{\partial W_{1}}{\partial n}\bigg\vert_{0}\right]+{\cal R}
\end{multline}
where ${\cal R}$ is the resulting power series and error remainder of the integration defined in terms of Bernoulli polynomials. Euler-Maclaurin integration is rarely convergent, and in this case serves only as an approximation within the domain where the error remainder is small and bounded; see Ref.~\cite{greiner2012thermodynamics} for the non-relativistic case. In this analysis, we keep the zeroth and first order terms in the Euler-Maclaurin formula. We note that regularization of the excess terms in \req{eulermaclaurin} is done in the context of strong field QED~\cite{greiner2008quantum} though that is outside our scope.

After truncation of the series and error remainder and combining \req{partitionpower:1} through \req{eulermaclaurin}, the partition function can then be written in terms of modified Bessel $K_{\nu}$ functions of the second kind, yielding
\begin{multline}
 \label{boltzmann}
 \ln{\cal Z}_{e^{+}e^{-}}\simeq\frac{T^{3}V}{\pi^{2}}\sum_{s'}^{\pm1}\left[\xi_{s'}\cosh{\frac{\mu}{T}}\right]\\
 \left(x_{s'}^{2}K_{2}(x_{s'})+\frac{b_{0}}{2}x_{s'}K_{1}(x_{s'})+\frac{b_{0}^{2}}{12}K_{0}(x_{s'})\right)\,,
\end{multline}
\begin{align}
    \label{xfunc}
    x_{s'}=\frac{{\tilde m}_{s'}}{T}=\sqrt{\frac{m_{e}^{2}}{T^{2}}+b_{0}\left(1-\frac{g}{2}s'\right)}\,.
\end{align}
The latter two terms in \req{boltzmann} proportional to $b_{0}K_{1}$ and $b_{0}^{2}K_{0}$ are the uniquely magnetic terms present containing both spin and Landau orbital influences in the partition function. The $K_{2}$ term is analogous to the free Fermi gas~\cite{greiner2012thermodynamics} being modified only by spin effects.

This \lq separation of concerns\rq\ can be rewritten as
\begin{align}
 \label{spin}
 \ln{\cal Z}_{\rm S}=\frac{T^{3}V}{\pi^{2}}\sum_{s'}^{\pm1}\left[\xi_{s'}\cosh{\frac{\mu}{T}}\right]\left(x_{s'}^{2}K_{2}(x_{s'})\right)\,,
\end{align}
\begin{multline}
 \label{spinorbit}
 \ln{\cal Z}_{\rm SO}=\frac{T^{3}V}{\pi^{2}}\sum_{s'}^{\pm}\left[\xi_{s'}\cosh{\frac{\mu}{T}}\right]\\
 \left(\frac{b_{0}}{2}x_{s'}K_{1}(x_{s'})+\frac{b_{0}^{2}}{12}K_{0}(x_{s'})\right)\,,
\end{multline}
where the spin (S) and spin-orbit (SO) partition functions can be considered independently. When the magnetic scale $b_{0}$ is small, the spin-orbit term \req{spinorbit} becomes negligible leaving only paramagnetic effects in \req{spin} due to spin. In the non-relativistic limit, \req{spin} reproduces a quantum gas whose Hamiltonian is defined as the free particle (FP) Hamiltonian plus the magnetic dipole (MD) Hamiltonian which span two independent Hilbert spaces ${\cal H}_{\rm FP}\otimes{\cal H}_{\rm MD}$.

Writing the partition function as \req{boltzmann} instead of \req{partitionpower:1} has the additional benefit that the partition function remains finite in the free gas $({\cal B}\rightarrow0)$ limit. This is because the free Fermi gas and \req{spin} are mathematically analogous to one another. As the Bessel $K_{\nu}$ functions are evaluated as functions of $x_{\pm}$ in \req{xfunc}, the \lq free\rq\ part of the partition $K_{2}$ is still subject to spin magnetization effects. In the limit where ${\cal B}\rightarrow0$, the free Fermi gas is recovered in both the Boltzmann approximation $k=1$ and the general case $\sum_{k=1}^{\infty}$.

%%%%%%%%%%%%%%%%%%%%%%%%%%%%%%%%%%%%%%%
\subsection{Chemical potential}
\label{sec:chem}
%%%%%%%%%%%%%%%%%%%%%%%%%%%%%%%%%%%%%%%
\noindent In presence of a magnetic field in the Boltzmann approximation, the charge neutrality condition \req{chargeneutrality} becomes
\begin{multline}
 \label{chem}
 \sinh\frac{\mu}{T}=n_{p}\frac{\pi^{2}}{T^{3}}\\
 \left[\sum_{s'}^{\pm1}\xi_{s'}\!\left(\!x_{s'}^{2}K_{2}(x_{s'})\!+\!\frac{b_{0}}{2}x_{s'}K_{1}(x_{s'})\!+\!\frac{b_{0}^{2}}{12}K_{0}(x_{s'}\!)\!\right)\!\right]^{-1}\!.
\end{multline}
\req{chem} is fully determined by the right-hand-side expression if the spin fugacity is set to unity $\eta=0$ implying no external bias to the number of polarizations except as a consequence of the difference in energy eigenvalues. In practice, the latter two terms in \req{chem} are negligible to chemical potential in the bounds of the primordial $e^{+}e^{-}$ plasma considered and only becomes relevant for extreme (see \rf{fig:chemicalpotential}) magnetic field strengths well outside our scope.

\req{chem} simplifies if there is no external magnetic field $b_{0}=0$ into
\begin{align}
    \label{simpchem:1}
    \sinh\frac{\mu}{T}=n_{p}\frac{\pi^{2}}{T^{3}}\left[2\cosh\frac{\eta}{T}\left(\frac{m_{e}}{T}\right)^{2}K_{2}\left(\frac{m_{e}}{T}\right)\right]^{-1}\,.
\end{align}

%%%%%%%%%%%%%%%%%%%%%%%%%%%%%%%%%%%%%%%
\begin{figure}[ht]
 \centering
 \includegraphics[width=0.45\textwidth]{plots/ChemicalPotential_06.png}
 \caption{The chemical potential over temperature $\mu/T$ is plotted as a function of temperature with differing values of spin potential $\eta$ and magnetic scale $b_{0}$.}
 \label{fig:chemicalpotential}
\end{figure}
%%%%%%%%%%%%%%%%%%%%%%%%%%%%%%%%%%%%%%%

In \rf{fig:chemicalpotential} we plot the chemical potential $\mu/T$ in \req{chem} and \req{simpchem:1} which characterizes the importance of the charged lepton asymmetry as a function of temperature. Since the baryon (and thus charged lepton) asymmetry remains fixed, the suppression of $\mu/T$ at high temperatures indicates a large pair density which is seen explicitly in \rf{fig:densityratio}. The black line corresponds to the $b_{0}=0$ and $\eta=0$ case. 

The para-diamagnetic contribution from \req{spinorbit} does not appreciably influence $\mu/T$ until the magnetic scales involved become incredibly large well outside the observational bounds defined in \req{igmf} and \req{tbscale} as seen by the dotted blue curves of various large values $b_{0}=\{25,\ 50,\ 100,\ 300\}$. The chemical potential is also insensitive to forcing by the spin potential until $\eta$ reaches a significant fraction of the electron mass $m_{e}$ in size. The chemical potential for large values of spin potential $\eta=\{100,\ 200,\ 300,\ 400,\ 500\}\ \keV$ are also plotted as dashed black lines with $b_{0}=0$.

It is interesting to note that there are crossing points where a given chemical potential can be described as either an imbalance in spin-polarization or presence of external magnetic field. While spin potential suppresses the chemical potential at low temperatures, external magnetic fields only suppress the chemical potential at high temperatures.

The profound insensitivity of the chemical potential to these parameters justifies the use of the free particle chemical potential (black line) in the ranges of magnetic field strength considered for cosmology. Mathematically this can be understood as $\xi$ and $b_{0}$ act as small corrections in the denominator of \req{chem} if expanded in powers of these two parameters.

%%%%%%%%%%%%%%%%%%%%%%%%%%%%%%%%%%%%%%%
\section{Electron-positron plasma magnetization}
\label{sec:magnetization}
%%%%%%%%%%%%%%%%%%%%%%%%%%%%%%%%%%%%%%%
\noindent The total magnetic flux within a region of space can be written as the sum of external fields and the magnetization of the medium via
\begin{align}
 \label{totalmag}
 {\cal B}_{\rm total} = {\cal B} + {\cal M}\,.
\end{align}
For the simplest mediums without ferromagnetic or hysteresis considerations, the relationship can be parameterized by the susceptibility $\chi$ of the medium as
\begin{align}
 \label{susceptibility}
 {\cal B}_{\rm total} = (1+\chi){\cal B}\,,\qquad {\cal M} = \chi{\cal B}\,,
\end{align}
with the possibility of both paramagnetic materials $(\chi>1)$ and diamagnetic materials $(\chi<1)$. The $e^{+}e^{-}$ plasma however does not so neatly fit in either category as given by \req{spin} and \req{spinorbit}. In general, the susceptibility of the gas will itself be a field dependant quantity given by
\begin{align}
    \chi \equiv \frac{\partial{\cal M}}{\partial{\cal B}}\,.
\end{align}

In our analysis, the external magnetic field always appears within the context of the magnetic scale $b_{0}$, therefore we can introduce the change of variables
\begin{align}
 \frac{\partial b_{0}}{\partial{\cal B}}=\frac{e}{T^{2}}\,.
\end{align}
The magnetization of the $e^{+}e^{-}$ plasma described by the partition function in \req{boltzmann} can then be written as
\begin{align}
 \label{defmagetization}
 {\cal M}\equiv\frac{T}{V}\frac{\partial}{\partial{\cal B}}\ln{{\cal Z}_{e^{+}e^{-}}} = \frac{T}{V}\left(\frac{\partial b_{0}}{\partial{\cal B}}\right)\frac{\partial}{\partial b_{0}}\ln{{\cal Z}_{e^{+}e^{-}}}\,,
\end{align}
Magnetization arising from other components in the cosmic gas (protons, neutrinos, etc.) could in principle also be included. Localized inhomogeneities of matter evolution are often non-trivial and generally be solved numerically using magneto-hydrodynamics (MHD)~\cite{melrose2008quantum,Vazza:2017qge,Vachaspati:2020blt}. In the context of MHD, primordial magnetogenesis from fluid flows in the electron-positron epoch was considered in~\cite{Gopal:2004ut,Perrone:2021srr}.

We introduce dimensionless units for magnetization ${\mathfrak M}$ by defining the critical field strength
\begin{align}
 {\cal B}_{C}\equiv\frac{m_{e}^{2}}{e}\,,\qquad{\mathfrak M}\equiv\frac{\cal M}{{\cal B}_{C}}\,.
\end{align}
The scale ${\cal B}_{C}$ is where electromagnetism is expected to become subject to non-linear effects, though luckily in our regime of interest, electrodynamics should be linear. We note however that the upper bounds of IGMFs in \req{igmf} (with $b_{0}=10^{-3}$; see \req{tbscale}) brings us to within $1\%$ of that limit for the external field strength in the temperature range considered.

The total magnetization ${\mathfrak M}$ can be broken into the sum of magnetic moment parallel ${\mathfrak M}_{+}$ and magnetic moment anti-parallel ${\mathfrak M}_{-}$ contributions
\begin{align}
\label{g2mag}
{\mathfrak M}={\mathfrak M}_{+}+{\mathfrak M}_{-}\,.
\end{align}
We note that the expression for the magnetization simplifies significantly for $g=2$ which is the \lq natural\rq\ gyro-magnetic factor~\cite{Evans:2022fsu,Rafelski:2022bsv} for Dirac particles. For illustration, the $g=2$ magnetization from \req{defmagetization} is then
\begin{align}
 \label{g2magplus}
 {\mathfrak M}_{+}&=\frac{e^{2}}{\pi^{2}}\frac{T^{2}}{m_{e}^{2}}\xi\cosh{\frac{\mu}{T}}\left[\frac{1}{2}x_{+}K_{1}(x_{+})+\frac{b_{0}}{6}K_{0}(x_{+})\right]\,,\\
\begin{split}  
 \label{g2magminus}
 -{\mathfrak M}_{-}&=\frac{e^{2}}{\pi^{2}}\frac{T^{2}}{m_{e}^{2}}\xi^{-1}\cosh{\frac{\mu}{T}}\\
 &\left[\left(\frac{1}{2}+\frac{b_{0}^{2}}{12x_{-}^{2}}\right)x_{-}K_{1}(x_{-})+\frac{b_{0}}{3}K_{0}(x_{-})\right]\,,
\end{split}\\
 x_{+}&=\frac{m_{e}}{T}\,,\qquad
 x_{-}=\sqrt{\frac{m_{e}^{2}}{T^{2}}+2b_{0}}\,.
\end{align}
As the $g$-factor of the electron is only slightly above two at $g\simeq2.00232$~\cite{Tiesinga:2021myr}, the above two expressions for ${\mathfrak M}_{+}$ and ${\mathfrak M}_{-}$ are only modified by a small amount because of anomalous magnetic moment (AMM) and would be otherwise invisible on our figures. We will revisit AMM in \rsec{sec:gfac}.

%%%%%%%%%%%%%%%%%%%%%%%%%%%%%%%%%%%%%%%
\begin{figure}[ht]
 \centering
 \includegraphics[width=0.48\textwidth]{plots/Magnetization_Hc_new004.png}
 \caption{The magnetization ${\mathfrak M}$, with $g=2$, of the primordial $e^{+}e^{-}$ plasma is plotted as a function of temperature.}
 \label{fig:magnet} 
\end{figure}
%%%%%%%%%%%%%%%%%%%%%%%%%%%%%%%%%%%%%%%

%%%%%%%%%%%%%%%%%%%%%%%%%%%%%%%%%%%%%%%
\section{Relativistic paramagnetism}
\label{sec:paramagnetism}
%%%%%%%%%%%%%%%%%%%%%%%%%%%%%%%%%%%%%%%
\noindent In \rf{fig:magnet}, we plot the magnetization as given by \req{g2magplus} and \req{g2magminus} with the spin potential set to unity $\xi=1$. The lower (solid red) and upper (solid blue) bounds for cosmic magnetic scale $b_{0}$ are included. The external magnetic field strength ${\cal B}/{\cal B}_{C}$ is also plotted for lower (dotted red) and upper (dotted blue) bounds. Since the derivative of the partition function governing magnetization may manifest differences between Fermi-Dirac and the here used Boltzmann limit more acutely, out of abundance of caution, we indicate extrapolation outside the domain of validity of the Boltzmann limit with dashes.

We see in \rf{fig:magnet} that the $e^{+}e^{-}$ plasma is overall paramagnetic and yields a positive overall magnetization which is contrary to the traditional assumption that matter-antimatter plasma lack significant magnetic responses of their own in the bulk. With that said, the magnetization never exceeds the external field under the parameters considered which shows a lack of ferromagnetic behavior. 

The large abundance of pairs causes the smallness of the chemical potential seen in~\rf{fig:chemicalpotential} at high temperatures. As the universe expands and temperature decreases, there is a rapid decrease of the density $n_{e^{\pm}}$ of $e^{+}e^{-}$ pairs. This is the primary the cause of the rapid paramagnetic decrease seen in \rf{fig:magnet} above $T=21\keV$. At lower temperatures $T<21\keV$ there remains a small electron excess (see~\rf{fig:densityratio}) needed to neutralize proton charge. These excess electrons then govern the residual magnetization and dilutes with cosmic expansion.

An interesting feature of \rf{fig:magnet} is that the magnetization in the full temperature range increases as a function of temperature. This is contrary to Curie's law~\cite{greiner2012thermodynamics} which stipulates that paramagnetic susceptibility of a laboratory material is inversely proportional to temperature. However, Curie's law applies to systems with fixed number of particles which is not true in our situation; see \rsec{sec:perlepton}.

A further consideration is possible hysteresis as the $e^{+}e^{-}$ density drops with temperature. It is not immediately obvious the gas's magnetization should simply \lq degauss\rq\ so rapidly without further consequence. If the very large paramagnetic susceptibility present for $T\simeq m_{e}$ is the origin of an overall magnetization of the plasma, the conservation of magnetic flux through the comoving surface ensures that the initial residual magnetization is preserved at a lower temperature by Faraday induced kinetic flow processes however our model presented here cannot account for such effects. Some consequences of enforced magnetization are considered in \rsec{sec:ferro}.

Early universe conditions may also apply to some extreme stellar objects with rapid change in $n_{e^{\pm}}$ with temperatures above $T=21\keV$. Production and annihilation of $e^{+}e^{-}$ plasmas is also predicted around compact stellar objects~\cite{Ruffini:2009hg,Ruffini:2012it} potentially as a source of gamma-ray bursts (GRB).

%%%%%%%%%%%%%%%%%%%%%%%%%%%%%%%%%%%%%%%
\begin{figure}[ht]
 \centering
 \includegraphics[width=0.45\textwidth]{plots/GFactor_05.png}
 \caption{The magnetization $\mathfrak M$ as a function of $g$-factor plotted for several temperatures with magnetic scale $b_{0}=10^{-3}$ and polarization fugacity $\xi=1$.}
 \label{fig:gfac} 
\end{figure}
%%%%%%%%%%%%%%%%%%%%%%%%%%%%%%%%%%%%%%%

%%%%%%%%%%%%%%%%%%%%%%%%%%%%%%%%%%%%%%%
\subsection{Dependency on g-factor}
\label{sec:gfac}
%%%%%%%%%%%%%%%%%%%%%%%%%%%%%%%%%%%%%%%

\noindent As discussed at the end of \rsec{sec:magnetization}, the AMM of $e^{+}e^{-}$ is not relevant in the present model. However out of academic interest, it is valuable to consider how magnetization is effected by changing the $g$-factor significantly.

The influence of AMM would be more relevant for the magnetization of baryon gasses since the $g$-factor for protons $(g\approx5.6)$ and neutrons $(g\approx3.8)$ are substantially different from $g=2$. The influence of AMM on the magnetization of thermal systems with large baryon content (neutron stars, magnetars, hypothetical bose stars, etc.) is therefore also of interest~\cite{Ferrer:2019xlr,Ferrer:2023pgq}.

\req{g2magplus} and \req{g2magminus} with arbitrary $g$ reintroduced is given by
\begin{multline}
    \label{arbg:1}
    {\mathfrak M}=\frac{e^{2}}{\pi^{2}}\frac{T^{2}}{m_{e}^{2}}\sum_{s'}^{\pm1}\xi_{s'}\cosh{\frac{\mu}{T}}\\
    \left[C^{1}_{s'}(x_{s'})K_{1}(x_{s'})+C^{0}_{s'}K_{0}(x_{s'})\right]\,,
\end{multline}
\begin{align}
    \label{arbg:2}
    C^{1}_{s'}(x_{\pm}) &= \left[\frac{1}{2}-\left(\frac{1}{2}-\frac{g}{4}s'\right)\left(1+\frac{b^2_0}{12x^{2}_{s'}}\right)\right]x_{s'}\,,\\
    C^{0}_{s'} &= \left[\frac{1}{6}-\left(\frac{1}{4}-\frac{g}{8}s'\right)\right]b_0\,.
\end{align}
where $x_{s'}$ was previously defined in \req{xfunc}.

In \rf{fig:gfac}, we plot the magnetization as a function of $g$-factor between $4>g>-4$ for temperatures $T=\{511,\ 300,\ 150,\ 70\}\keV$. We find that the magnetization is sensitive to the value of AMM revealing a transition point between paramagnetic $({\mathfrak M}>0)$ and diamagnetic gasses $({\mathfrak M}<0)$. 

Curiously, the transition point was numerically determined to be around $g\simeq1.1547$ in the limit $b_{0}\rightarrow0$. The exact position of this transition point however was found to be both temperature and $b_{0}$ sensitive, though it moved little in the ranges considered.

It is not surprising for there to be a transition between diamagnetism and paramagnetism given that the partition function (see \req{spin} and \req{spinorbit}) contained elements of both. With that said, the transition point presented at $g\approx1.15$ should not be taken as exact because of the approximations used to obtain the above results. 

It is likely that the exact transition point has been altered by our taking of the Boltzmann approximation and Euler-Maclaurin integration steps. It is known that the Klein-Gordon-Pauli solutions to the Landau problem in \req{kgp} have periodic behavior~\cite{Steinmetz:2018ryf,Evans:2022fsu,Rafelski:2022bsv} for $|g|=k/2$ (where $k\in1,2,3\ldots$).

These integer and half-integer points represent when the two Landau towers of orbital levels match up exactly. Therefore, we propose a more natural transition between the spinless diamagnetic gas of $g=0$ and a paramagnetic gas is $g=1$. A more careful analysis is required to confirm this, but that our numerical value is close to unity is suggestive.

%%%%%%%%%%%%%%%%%%%%%%%%%%%%%%%%%%%%%%%
\subsection{Magnetization per lepton}
\label{sec:perlepton}
%%%%%%%%%%%%%%%%%%%%%%%%%%%%%%%%%%%%%%%
\noindent Despite the relatively large magnetization seen in \rf{fig:magnet}, the average contribution per lepton is only a small fraction of its overall magnetic moment indicating the magnetization is only loosely organized. Specifically, the magnetization regime we are in is described by
\begin{align}
 \label{fractionalmagnetization}
 {\cal M}\ll\mu_{B}\frac{N_{e^{+}}+N_{e^{-}}}{V}\,,\qquad\mu_{B}\equiv\frac{e}{2m_{e}}\,,
\end{align}
where $\mu_{B}$ is the Bohr magneton and $N=nV$ is the total particle number in the proper volume V. To better demonstrate that the plasma is only weakly magnetized, we define the average magnetic moment per lepton given by along the field ($z$-direction) axis as
\begin{align}
 \label{momentperlepton}
 \vert\vec{m}\vert_{z}\equiv\frac{{\cal M}}{n_{e^{-}}+n_{e^{+}}}\,,\qquad\vert\vec{m}\vert_{x}=\vert\vec{m}\vert_{y}=0\,.
\end{align}
Statistically, we expect the transverse expectation values to be zero. We emphasize here that despite $|\vec{m}|_{z}$ being nonzero, this doesn't indicate a nonzero spin angular momentum as our plasma is nearly matter-antimatter symmetric. The quantity defined in \req{momentperlepton} gives us an insight into the microscopic response of the plasma.

%%%%%%%%%%%%%%%%%%%%%%%%%%%%%%%%%%%%%%%
\begin{figure}[ht]
 \centering
 \includegraphics[width=0.45\textwidth]{plots/NewMagnetizationDensity004_Boltz.png}
 \caption{The magnetic moment per lepton $\vert\vec{m}\vert_{z}$ along the field axis as a function of temperature.}
 \label{fig:momentperlepton}
\end{figure}
%%%%%%%%%%%%%%%%%%%%%%%%%%%%%%%%%%%%%%%

The average magnetic moment $\vert\vec{m}\vert_{z}$ defined in \req{momentperlepton} is plotted in \rf{fig:momentperlepton} which displays how essential the external field is on the \lq per lepton\rq\ magnetization. Both the $b_{0}=10^{-11}$ (lower plot, red curve) and $b_{0}=10^{-3}$ (upper plot, blue curve) cosmic magnetic scale bounds are plotted in the Boltzmann approximation. The dashed lines indicate where this approximation is only qualitatively correct. For illustration, a constant magnetic field case (solid green line) with a comoving reference value chosen at temperature $T_{0}=10\keV$ is also plotted.

If the field strength is held constant, then the average magnetic moment per lepton is suppressed at higher temperatures as expected for magnetization satisfying Curie's law. The difference in \rf{fig:momentperlepton} between the non-constant (red and blue solid curves) case and the constant field (solid green curve) case demonstrates the importance of the conservation of primordial magnetic flux in the plasma, required by \req{bscale}.

While not shown, if \rf{fig:momentperlepton} was extended to lower temperatures, the magnetization per lepton of the constant field case would be greater than the non-constant case which agrees with our intuition that magnetization is easier to achieve at lower temperatures. This feature again highlights the importance of flux conservation in the system and the uniqueness of the primordial cosmic environment.

%%%%%%%%%%%%%%%%%%%%%%%%%%%%%%%%%%%%%%%
\section{Spin potential and ferromagnetism}
\label{sec:ferro}
\noindent Up to this point, we have neglected the impact that a nonzero spin potential $\eta\neq0$ (and thus $\xi\neq1$) would have on the primordial $e^{+}e^{-}$ plasma magnetization. In the limit that $(m_{e}/T)^2\gg b_0$ the magnetization given in \req{arbg:1} and \req{arbg:2} is entirely controlled by the spin fugacity $\xi$ asymmetry generated by the spin potential $\eta$ yielding up to first order ${\cal O}(b_{0})$ in magnetic scale
\begin{multline}
 \label{ferro}
 \lim_{m_{e}^{2}/T^{2}\gg b_0}{\mathfrak M}=\frac{g}{2}\frac{e^{2}}{\pi^{2}}\frac{T^{2}}{m_{e}^{2}}\sinh{\frac{\eta}{T}}\cosh{\frac{\mu}{T}}\left[\frac{m_{e}}{T}K_{1}\left(\frac{m_{e}}{T}\right)\right]\\
 +b_{0}\left(g^{2}-\frac{4}{3}\right)\frac{e^{2}}{8\pi^{2}}\frac{T^{2}}{m_{e}^{2}}\cosh{\frac{\eta}{T}}\cosh{\frac{\mu}{T}}K_{0}\left(\frac{m_{e}}{T}\right)
 +{\cal O}\left(b_{0}^{2}\right)
\end{multline}

Given \req{ferro}, we can understand the spin potential as a kind of \lq ferromagnetic\rq\ influence on the primordial gas which allows for magnetization even in the absence of external magnetic fields. This interpretation is reinforced by the fact the leading coefficient is $g/2$.

We suggest that a variety of physics could produce a small nonzero $\eta$ within a domain of the gas. Such asymmetries could also originate statistically as while the expectation value of free gas polarization is zero, the variance is likely not.

As $\sinh{\eta/T}$ is an odd function, the sign of $\eta$ also controls the alignment of the magnetization. In the high temperature limit \req{ferro} with strictly $b_{0}=0$ assumes a form of to lowest order for brevity
\begin{align}
 \label{hiTferro}
 \lim_{m_{e}/T\rightarrow0}{\mathfrak M}\vert_{b_{0}=0}=\frac{g}{2}\frac{e^{2}}{\pi^{2}}\frac{T^{2}}{m_{e}^{2}}\frac{\eta}{T}\,,
\end{align}

While the limit in \req{hiTferro} was calculated in only the Boltzmann limit, it is noteworthy that the high temperature (and $m\rightarrow0$) limit of Fermi-Dirac distributions only differs from the Boltzmann result by a proportionality factor. 

The natural scale of the $e^{+}e^{-}$ magnetization with only a small spin fugacity ($\eta<1\eV$) fits easily within the bounds of the predicted magnetization during this era if the IGMF measured today was of primordial origin. The reason for this is that the magnetization seen in \req{g2magplus}, \req{g2magminus} and \req{ferro} are scaled by $\alpha{\cal B}_{C}$ where $\alpha$ is the fine structure constant.

%%%%%%%%%%%%%%%%%%%%%%%%%%%%%%%%%%%%%%%
\subsection{Self-magnetization}
\label{sec:self}
%%%%%%%%%%%%%%%%%%%%%%%%%%%%%%%%%%%%%%%

\noindent One exploratory model we propose is to fix the spin polarization asymmetry, described in \req{spotential}, to generate a homogeneous magnetic field which dissipates as the universe cools down. In this model, there is no external primordial magnetic field $({\cal B}_{\rm PMF}=0)$ generated by some unrelated physics, but rather the $e^{+}e^{-}$ plasma itself is responsible for the field by virtue of spin polarization.

%%%%%%%%%%%%%%%%%%%%%%%%%%%%%%%%%%%%%%%
\begin{figure}[ht]
 \centering
 \includegraphics[width=0.45\textwidth]{plots/Spinchemical_03.png}
 \caption{The spin potential $\eta$ and chemical potential $\mu$ are plotted under the assumption of self-magnetization through a nonzero spin polarization in bulk of the plasma.}
 \label{fig:self} 
\end{figure}
%%%%%%%%%%%%%%%%%%%%%%%%%%%%%%%%%%%%%%%

This would obey the following assumption of
\begin{align}
 \label{selfmag}
 {\mathfrak M}(b_{0})=\frac{{\cal M}(b_0)}{{\cal B}_{C}}\longleftrightarrow\frac{\cal B}{{\cal B}_{C}}=b_{0}\frac{T^{2}}{m_{e}^{2}}\,,
\end{align}
which sets the total magnetization as a function of itself. The spin polarization described by $\eta\rightarrow\eta(b_{0},T)$ then becomes a fixed function of the temperature and magnetic scale. The underlying assumption would be the preservation of the homogeneous field would be maintained by scattering within the gas (as it is still in thermal equilibrium) modulating the polarization to conserve total magnetic flux.

The result of the self-magnetization assumption in \req{selfmag} for the potentials is plotted in \rf{fig:self}. The solid lines indicate the curves for $\eta/T$ for differing values of $b_{0}=\{10^{-11},\ 10^{-7},\ 10^{-5},\ 10^{-3}\}$ which become dashed above $T=300\keV$ to indicate that the Boltzmann approximation is no longer appropriate though the general trend should remain unchanged. 

The dotted lines are the curves for the chemical potential $\mu/T$. At high temperatures we see that a relatively small $\eta/T$ is needed to produce magnetization owing to the large densities present. \rf{fig:self} also shows that the chemical potential does not deviate from the free particle case until the spin polarization becomes sufficiently high which indicates that this form of self-magnetization would require the annihilation of positrons to be incomplete even at lower temperatures.

This is seen explicitly in~\rf{fig:polarswap} where we plot the numerical density of particles as a function of temperature for spin aligned $(+\eta)$ and spin anti-aligned $(-\eta)$ species for both positrons $(-\mu)$ and electrons $(+\mu)$. Various self-magnetization strengths are also plotted to match those seen in~\rf{fig:self}. The nature of $T_{\rm split}$ changes under this model where polarization states are extinguished rather than antimatter states.

%%%%%%%%%%%%%%%%%%%%%%%%%%%%%%%%%%%%%%%
\begin{figure}[ht]
 \centering
 \includegraphics[width=0.45\textwidth]{plots/NumeberDensitySpin.png}
 \caption{The number density $n_{e^{\pm}}$ of polarized electrons (top) and positrons (bottom) under the self-magnetization model for differing values of $b_{0}$}
 \label{fig:polarswap} 
\end{figure}
%%%%%%%%%%%%%%%%%%%%%%%%%%%%%%%%%%%%%%%

%%%%%%%%%%%%%%%%%%%%%%%%%%%%%%%%%%%%%%%
\subsection{Matter inhomogeneities in the cosmic plasma}
\label{sec:inhomogeneous}
%%%%%%%%%%%%%%%%%%%%%%%%%%%%%%%%%%%%%%%
\noindent In general, an additional physical constraint is required to fully determine $\mu$ and $\eta$ simultaneously as both potentials have mutual dependency (see \rsec{sec:ferro}). We note that spin polarizations are not required to be in balanced within a single species to preserve angular momentum. For magnetized domains or finite volumes, boundary/surface conditions would also need to be considered.

Measurements of the CMB~\cite{Planck:2018vyg} indicates that the early universe was home to domains of slightly higher and lower baryon densities which resulted in the presence of galactic super-clusters, cosmic filaments, and great voids seen today. However, the CMB, as measured today, is blind to the localized inhomogeneities required for gravity to begin galaxy and supermassive black hole formation. Such acute inhomogeneities distributed like a dust~\cite{Grayson:2023flr} in the plasma would make the proton density sharply and spatially dependant $n_{p}\rightarrow n_{p}(x)$ which would directly affect the potentials $\mu(x)$ and $\eta(x)$ and thus the density of electrons and positrons locally. This suggests that $e^{+}e^{-}$ may play a role in the initial seeding of gravitationally collapsing systems.

If the primordial plasma were home to such small localized magnetic domains, the associated nonzero local angular momentum within these domains would provide a natural mechanism for the formation of rotating galaxies today. Recent measurements by the James Webb Space Telescope (JWST)~\cite{Yan:2022sxd,adams2023discovery,arrabal2023spectroscopic} indicate that galaxy formation began surprisingly early at large redshift values of $z\gtrsim10$ within the first 500 million years of the universe requiring gravitational collapse to begin in a hotter environment than expected. Additionally the observation of supermassive black holes already present~\cite{CEERSTeam:2023qgy} in this same high redshift period (with millions of solar masses) indicates the need for exceptionally local high density regions in the early universe whose generation is not yet explained and likely need to exist long before the recombination epoch.

%%%%%%%%%%%%%%%%%%%%%%%%%%%%%%%%%%%%%%%
\section{Summary and Outlook}
\label{sec:conclusions}
\noindent In this work, we explored the paramagnetic magnetization of the $e^{+}e^{-}$ thermal Fermi gas in the temperature range of the early universe between $2000\keV>T>20\keV$ expanding on our work in~\cite{Rafelski:2023emw}. The combination of strong magnetic fields, high matter-antimatter density, and relatively high temperatures (far higher than the Sun's core temperature~\cite{Bahcall:2000nu} of $T_{\odot}=1.37\keV$) makes this era of the universe unique in cosmology and astrophysics.

We determined that the electron/positron-to-baryon density ratio before BBN was $n_{e^{\pm}}/n_{B}=4.47\times10^{8}$. We showed that the primordial universe $e^{+}e^{-}$ plasma has paramagnetic properties when subjected to an external field. Our analysis shows this paramagnetism does not diminish in the distant past with high temperature unlike most magnetic phenomenon, but is enhanced due to
\begin{itemize}
 \item[a.] the high density of electron-positron pairs which exist in the plasma during this era and
 \item[b.] the conservation of magnetic flux in an expanding universe which helps maintain the polarization of the plasma.
\end{itemize}
Despite this, the plasma is only ever weakly magnetized as most of the spins are highly disorganized.

We introduced a statistical potential which characterizes the spin polarization within the gas. This model was then used to analyze the required polarization necessary to match the expected magnetic flux of the era if today's IGMF could be extrapolated. While this is an incomplete model, it does demonstrate that the early universe only required a small asymmetry in spin polarization to produce significant magnetic fields. We suggest further efforts to connect the bulk magnetization of a polarized gas to Amp{\`e}rian magnetization generated through currents and inhomogeneous flows. 

Further extensions would be to improve the high temperature domain by computing the full partition function instead of the Boltzmann approximated form. At higher temperatures, more particles such as neutrinos would need to be included as they become thermally coupled. Adding in the baryons such as protons and neutrons may also be needed. Even in a universe with overall zero angular momentum, tiny localized imbalances in angular momentum in the primordial era likely consequences on observed large-scale structure today. 

Recent measurements by the JWST indicate that large-scale structure, galactic, and supermassive black hole formation occurred earlier than expected which may indicate unusual matter agglomeration in even primordial eras. As the chemical potential of the $e^{+}e^{-}$ plasma is sensitive to the baryon density, any local spatial variations can effect magnetization substantially. Spin polarization domains may then provide the necessary seeds to generate collapsing (and rotating) cosmic structure and galaxies.

%%%%%%%%%%%%%%%%%%%%%%%%%%%%%%%%%%%%%%%
\acknowledgments
\label{sec:ack}
%%%%%%%%%%%%%%%%%%%%%%%%%%%%%%%%%%%%%%%
\noindent Johann Rafelski would like to acknowledge the fruitful discussions with Massimo Giovannini at CERN which partly inspired this work. 

%%%%%%%%%%%%%%%%%%%%%%%%%%%%%%%%%%%%%%%
\bibliographystyle{unsrtnat}
\bibliography{refs-plasma-partition}
%%%%%%%%%%%%%%%%%%%%%%%%%%%%%%%%%%%%%%%

\end{document}
