
This dense plasma environment is where BBN occurred and where similar plasmas can still be found within exotic stars such as magnetar~\cite{Broderick:2000pe}s. The contemporary relic magnetic fields may then be an artifact of this final time of Universe-scale magnetization in a manner similar to how the CMB is a relic of the time of charge recombination.

We can now connect back to the consideration of cosmic magnetic fields as they might have risen in the environment of early Universe plasmas noting that such primordial magnetic fields would be lensed through each of the various plasmas that existed when the Universe was far hotter and denser. As magnetic flux is conserved over co-moving surfaces, we see in \rf{relic_plot} that the primordial relic field is expected to dilute as $B\propto1/a(t)^{2}$. This means the contemporary small bounded values of $5\times10^{-12}\ \mathrm{T}>B_{relic}>10^{-20}\ \mathrm{T}$ (coherent over $\mathcal{O}(1\ \mathrm{Mpc})$ distances) may have once represented large magnetic fields in the early Universe. This is relic magnetic field would then be generated by the last phase of significant magnetization in the early Universe. This figure is meant to be illustrative and it is unlikely the magnetization of the Universe would proceed unhindered and unaltered into the ultra-dense plasma phases of the early Universe. Of particular interest to us is the electron-positron plasma which existed in the early Universe especially at temperatures $T>35\ \mathrm{keV}$ which was the last matter(antimatter) plasma in the Universe where the energy its density exceeded that of the proton/neutron baryon energy density.